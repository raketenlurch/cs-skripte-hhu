% !TEX root = /home/annki/cs-skripte-hhu/Informatik/Pflichtmodule/einführung-rechnernetze-datenbanken-betriebssysteme-rdb/Skript.tex
\documentclass[a4paper,10pt]{article}
\usepackage[utf8]{inputenc}
\usepackage[ngerman]{babel}
\usepackage{amsmath,amssymb,amstext}

\title{Ausführliches Skript zur Vorlesung Einführung Rechnernetze, Datenbanken, Betriebssysteme}
\author{Frieda Nord}
\date{Sommersemester 2020}

\begin{document}
  \maketitle
  \tableofcontents

  \section{Datendarstellung}
  Um darüber sprechen zu können, wie Netzwerke funktionieren, die Daten übertragen,
  ist zunächst ein Grundverständnis dafür notwendig, wie Daten überhaut aussehen.
  Deshalb werden wir zunächst besprechen, wie eine digitale Datendarstellung
  funktioniert. Digitalrechner speichern Daten in Binärform. Mit Binärform sind
  Folgen von Nullen und Einsen gemeint. Daten, die auf Rechnern gespeichert oder
  in einem Netzwerk übertragen werden sollen, müssen also durch Nullen und Einsen
  ausgedrückt werden.

  %\subsection{Dezimal- und Binärzahlen}
  %\paragraph{Dezimalsystem}\mbox{}\\
  %Das Dezimalsystem, welches auch als Zehnersystem oder oder dekadisches System
  %bezeichnet wird, ist ein Zahlensystem, das als Basis die Zahl 10 verwendet. Im
  %Dezimalsystem haben die einzelnen Stellen der Zahl (von rechts nach links) die
  %Wertigkeiten 1, 10, 100, 1000,... . Um jede beliebige Zahl darstellen zu können
  %benötigt man deswegen zehn verschiedene Ziffern (0, ..., 9). Wenn insgesamt
  %drei Stellen zur Verfügung stehen können die Zahlen von 000 bis 999 dargestellt
  %werden, ein dreistelliger \glqq Speicher\grqq{} könnte als von $10^3 = 1000$
  %verschiedenen Zahlen enthalten. Mit vier Zahlen lassen sich dementsprechend
  %die Zahlen von 0000 bis 9999 ($10^4 = 10000$ verschiedene Zahlen) darstellen.

  %\paragraph{Binärsystem}\mbox{}\\
  %Das Binärsystem, welches auch Dualsystem oder Zweiersystem genannt wird, ist
  %ein Zahlensystem zur Basis 2.
  \subsection{Dezimalsystem}
  Das Dezimalsystem, welches auch als Zehnersystem oder dekadisches System bezeichnet
  wird, ist ein Zahlensystem, das als Basis die Zahl 10 verwendet. Im Dezimalsystem
  haben die einzelnen Stellen der Zahl (von rechts nach links) die Wertigkeiten
  1, 10, 100, 1000, ... . Um jede beliebige Zahl darstellen zu können benötigt
  man deswegen zehn verschiedene Ziffern (0, ..., 9). Wenn insgesamt drei Stellen
  zur Verfügung stehen können die Zahlen von 000 bis 999 dargestellt werden, ein
  dreistelliger \glqq Speicher \grqq{} könnte als von $10^3 = 1000$ verschiedene
  Zahlen enthalten. Mit vier Zahlen lassen sich dementsprechend die Zahlen von
  0000 bis 9999 ($10^4 = 10000$ verschiedene Zahlen) darstellen.

  \subsection{Binärsystem}
  Das Binärsystem, welches auch Dualsystem oder Zweiersystem genannt wird, ist
  ein Zahlensystem zur Basis 2. Genau wie im Dezimalsystem haben auch im Binärsystem
  die Stellen (von rechts nach links) verschiedene Wertigkeiten (1, 2, 4, 8, 16,
  ...). 

  \paragraph{Addition}\mbox{}\\
  \paragraph{Subtraktion}\mbox{}\\
  \paragraph{Multiplikation}\mbox{}\\
  \paragraph{Division}\mbox{}\\
\end{document}
