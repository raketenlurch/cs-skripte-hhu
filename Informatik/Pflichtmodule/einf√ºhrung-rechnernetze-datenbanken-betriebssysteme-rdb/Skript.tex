% !TEX root = /home/annki/cs-skripte-hhu/Informatik/Pflichtmodule/einführung-rechnernetze-datenbanken-betriebssysteme-rdb/Skript.tex
\documentclass[a4paper,10pt]{article}
\usepackage[utf8]{inputenc}
\usepackage[ngerman]{babel}
\usepackage{amsmath,amssymb,amstext}

\title{Ausführliches Skript zur Vorlesung Einführung Rechnernetze, Datenbanken, Betriebssysteme}
\author{Frieda Nord}
\date{Sommersemester 2020}

\begin{document}
  \maketitle
  \tableofcontents

  \section{Datendarstellung}
  Um darüber sprechen zu können, wie Netzwerke funktionieren, die Daten übertragen,
  ist zunächst ein Grundverständnis dafür notwendig, wie Daten überhaut aussehen.
  Deshalb werden wir zunächst besprechen, wie eine digitale Datendarstellung
  funktioniert. Digitalrechner speichern Daten in Binärform. Mit Binärform sind
  Folgen von Nullen und Einsen gemeint. Daten, die auf Rechnern gespeichert oder
  in einem Netzwerk übertragen werden sollen, müssen also durch Nullen und Einsen
  ausgedrückt werden.

  \subsection{Dezimal- und Binärzahlen}
  \paragraph{Dezimalsystem}\mbox{}\\
  Das Dezimalsystem, welches auch als Zehnersystem oder oder dekadisches System
  bezeichnet wird, ist ein Zahlensyste, das als Basis die Zahl 10 verwendet.
  \paragraph{Binärsystem}
\end{document}
